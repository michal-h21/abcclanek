\documentclass{article}
\usepackage[utf8]{inputenc}
\usepackage[T1]{fontenc}
\usepackage[czech]{babel}
\usepackage{listings}
\usepackage{hyperref}
\usepackage[numberedsection=nolabel, toc]{glossaries}
\makeglossary
\newglossaryentry{tex4ht}{name=tex4ht, description={Konverzní nástroj pro převod \TeX ových dokumentů do jiných formátů, mimo jiné html nebo odt. \href{Homepage}{http://www.tug.org/applications/tex4ht/mn.html}}}
\usepackage[style=authoryear, backend = biber,sortlocale=cs_CZ]{biblatex}
\addbibresource{clanek.bib}
	\printbibliography
\begin{document}
\title{Pracujeme s TeX4ht}
\author{Michal Hoftich}
\maketitle
\tableofcontents
\section{Úvod}
V \href{http://www.abclinuxu.cz/clanky/tex-5-priklad-makra-pro-generovani-html}{článku na Abclinuxu.cz} 
pan \textcite{Olsak13}  popisuje konverzi \TeX ového souboru na html. 
Pod článkem se rozeběhla zajímavá debata nejen o nejlepším způsobu konverze
(La)\TeX u, ale také obecně o tom jestli má smysl psát dokumenty v \TeX u, nebo spíše v nělterém značkovacím jazyce, ať již je to některý dialekt xml, nebo třeba Markdown.

Protože mám jisté zkušenosti s používáním \gls{tex4ht}%
\footnote{Tvořím jednoduchý build systém \href{make4ht}{https://github.com/michal-h21/make4ht} 
a od něj odvozený \href{tex4ebook}{https://github.com/michal-h21/tex4ebook} 
pro přímou konverzi \TeX u do formátů pro čtečky knih},
rozhodl jsem se vytvořit krátký článek o jeho používání s češtinou a kódováním utf-8.

\section{Proč TeX4ht?}

\gls{tex4ht} se vyvíjí od druhé poloviny 90. let, v podstatě jako dílo 
jediného autora, Eitana Gurariho

\printglossary
\printbibliography
\end{document}

