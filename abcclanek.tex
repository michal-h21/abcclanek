\documentclass{article}
\usepackage[utf8]{inputenc}
\usepackage[T1]{fontenc}
\usepackage[czech]{babel}
\usepackage{csquotes}
\DeclareQuoteStyle{czech}
  {\quotedblbase}
	{\textquotedblleft}
	{\textquoteleft}
	{\textquoteright}
\AtBeginDocument{\shorthandoff{-"}}
\usepackage{minted}
\usepackage[style=authoryear, backend = biber,sortlocale=cs_CZ]{biblatex}
\usepackage{hyperref}
\usepackage[numberedsection=nolabel, toc]{glossaries}
\makeglossary
\newglossaryentry{tex4ht}{name=TeX4ht, description={Konverzní nástroj pro převod \TeX ových dokumentů do jiných formátů, mimo jiné html nebo odt. \href{http://www.tug.org/applications/tex4ht/mn.html}{Homepage}}}
\newglossaryentry{s4ht}{name={Soubory .4ht},text={.4ht}, description={Konfigurační soubory pro LaTeXové balíčky}}
\newglossaryentry{tex4ht-post}{name=tex4ht, 
	description={Postprocesor, který převádí dvi soubor na výstupní xml soubory}}
\newglossaryentry{t4ht}{name=t4ht,
	description={Další postprocesor, který na základě instrukcí předaných tex4ht vytváří CSS soubory a případně provádí konverzi obrázků.}}
%\newglossaryentry{s4ht}{name={Soubory 4ht}, description={Soubory s konfiguracemi balíčků pro TeX4ht}}

\addbibresource{clanek.bib}
\begin{document}
\title{Pracujeme s TeX4ht}
\author{Michal Hoftich}
\maketitle
\tableofcontents
\section{Úvod}
V \href{http://www.abclinuxu.cz/clanky/tex-5-priklad-makra-pro-generovani-html}{článku na Abclinuxu.cz} 
pan \textcite{Olsak13}  popisuje konverzi \TeX ového souboru na html. 
Pod článkem se rozeběhla zajímavá debata nejen o nejlepším způsobu konverze
(La)\TeX u, ale také obecně o tom jestli má smysl psát dokumenty v \TeX u, nebo spíše v nělterém značkovacím jazyce, ať již je to některý dialekt xml, nebo třeba Markdown.

Protože mám jisté zkušenosti s používáním \gls{tex4ht}%
\footnote{Tvořím jednoduchý build systém \href{https://github.com/michal-h21/make4ht}{make4ht} 
a od něj odvozený \href{https://github.com/michal-h21/tex4ebook}{tex4ebook} 
pro přímou konverzi \TeX u do formátů pro čtečky knih},
rozhodl jsem se vytvořit krátký článek o jeho používání s češtinou a kódováním utf-8. V článku také ukážu použití některých balíčků, například \verb|biblatex|,
\verb|glossaries|, nebo \verb|minted|.

Zdrojové kódy tohoto článku jsou dostupné na \href{https://github.com/michal-h21/abcclanek}{Githubu}.

\section{Proč TeX4ht?}

\gls{tex4ht} se vyvíjí od druhé poloviny 90. let, v podstatě jako dílo 
jediného autora, Eitana Gurariho. Tento autor bohužel před několika lety zemřel
a v současnosti patrně není nikdo, kdo by se v kódu plně orientoval 
a vývoj mírně řečeno stagnuje. 

Na druhou stranu, v současnosti je nejlepším konvertorem z 
\LaTeX u a poskytuje funkční infrastrukturu pro konfigurací libovolných maker. 
V současnosti je na tvůrcích balíčků nebo jejich uživatelích, aby poskytovali 
tyto konfigurace, není v reálných silách současných správců, aby tvořili konfigurace pro všechny balíčky.

\subsection{Na co konfigurace?}

Bez konfigurací bychom nezískali logické značkování, postprocesor \gls{tex4ht-post} sice vytvoří výstupní soubory, ale ty by obsahovaly jen vizuální formátování získané z dvi souboru, bez jakékoliv logické struktury
(nadpisy, seznamy, odkazy).

Proto \gls{tex4ht} obsahuje mechanismus \gls{s4ht} souborů, které jsou 
automaticky nahrávané spolu s balíčkem téhož jména a kde jsou umístěné 
redefinice některých maker, aby bylo možné je konfigurovat. Více informací 
v sekci \ref{sec:config}

\section{Proč }

\section{Čeština}

Mejme hypotetický článek pro Xe\LaTeX, který se budeme snažit přeložit s \gls{tex4ht}:

\inputminted{latex}{priklady/priklad1.tex}

\gls{tex4ht} obsahuje množství skriptů pro překlad dokumentů. Nejběžnější 
je \verb|htlatex|, který překládá dokument čístým \LaTeX em. Protože máš 
dokument je určený pro Xe\LaTeX, můžeme zkusit variantu tohoto skriptu,
\verb|htxelatex|. 

\begin{minted}{bash}
htxelatex priklad1
\end{minted}

Kompilace se zastaví s chybovou hláškou

\begin{verbatim}
	! LaTeX Error: Command `\acute' already defined in `'.
\end{verbatim}

To nám naznačuje, že něco není v pořádku. Podobných hlášek se objeví více,
můžete zmáčkout klávesu \verb|r| a \verb|enter| pro pokračování kompilace bez 
přerušení. Kompilační skripty překládají dokument \(3 \times\). 
Je to kvůli správnému fungování odkazů, \gls{tex4ht} ukládá při každém průchodu
informace o odkazech do pomocných souorů, tento mechanismus funguje
podobně jako \verb|\label - \ref| mechanismus v \LaTeX u.
\section{Konfigurace balíčků}\label{sec:config}

\printglossary
\printbibliography
\end{document}

