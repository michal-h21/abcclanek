\documentclass{article}
\usepackage[utf8]{inputenc}
\usepackage[T1]{fontenc}
\usepackage[czech]{babel}
\usepackage{minted}
\usepackage{hyperref}
\usepackage[numberedsection=nolabel, toc]{glossaries}
\makeglossary
\newglossaryentry{tex4ht}{name=TeX4ht, description={Konverzní nástroj pro převod \TeX ových dokumentů do jiných formátů, mimo jiné html nebo odt. \href{http://www.tug.org/applications/tex4ht/mn.html}{Homepage}}}
\usepackage[style=authoryear, backend = biber,sortlocale=cs_CZ]{biblatex}
\addbibresource{clanek.bib}
	\printbibliography
\begin{document}
\title{Pracujeme s TeX4ht}
\author{Michal Hoftich}
\maketitle
\tableofcontents
\section{Úvod}
V \href{http://www.abclinuxu.cz/clanky/tex-5-priklad-makra-pro-generovani-html}{článku na Abclinuxu.cz} 
pan \textcite{Olsak13}  popisuje konverzi \TeX ového souboru na html. 
Pod článkem se rozeběhla zajímavá debata nejen o nejlepším způsobu konverze
(La)\TeX u, ale také obecně o tom jestli má smysl psát dokumenty v \TeX u, nebo spíše v nělterém značkovacím jazyce, ať již je to některý dialekt xml, nebo třeba Markdown.

Protože mám jisté zkušenosti s používáním \gls{tex4ht}%
\footnote{Tvořím jednoduchý build systém \href{https://github.com/michal-h21/make4ht}{make4ht} 
a od něj odvozený \href{https://github.com/michal-h21/tex4ebook}{tex4ebook} 
pro přímou konverzi \TeX u do formátů pro čtečky knih},
rozhodl jsem se vytvořit krátký článek o jeho používání s češtinou a kódováním utf-8. V článku také ukážu použití některých balíčků, například \verb|biblatex|,
\verb|glossaries|, nebo \verb|minted|.

Zdrojové kódy tohoto článku jsou dostupné na \href{https://github.com/michal-h21/abcclanek}{Githubu}.

\section{Proč TeX4ht?}

\gls{tex4ht} se vyvíjí od druhé poloviny 90. let, v podstatě jako dílo 
jediného autora, Eitana Gurariho. Tento autor bohužel před několika lety zemřel
a v současnosti patrně není nikdo, kdo by se v kódu plně orientoval a vývoj poněkud stagnuje. 

Na druhou stranu, v současnosti je nejlepším konvertorem z 
\LaTeX u a poskytuje funkční infrastrukturu pro konfigurací libovolných maker.

\section{Čeština}

Mejme hypotetický článek pro Xe\LaTeX, který se budeme snažit přeložit s \gls{tex4ht}:

\inputminted{latex}{priklady/priklad1.tex}



\printglossary
\printbibliography
\end{document}

