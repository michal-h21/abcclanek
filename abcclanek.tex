\documentclass{article}
\usepackage[utf8]{inputenc}
\usepackage[T1]{fontenc}
\usepackage[czech]{babel}
\usepackage{newunicodechar}
\newunicodechar{fi}{fi}
\newunicodechar{ffi}{ffi}
\usepackage{csquotes}
\DeclareQuoteStyle{czech}
  {\quotedblbase}
	{\textquotedblleft}
	{\textquoteleft}
	{\textquoteright}
\AtBeginDocument{\shorthandoff{-"}}
\usepackage{minted}
\usepackage[style=authoryear, backend = biber,sortlocale=cs_CZ]{biblatex}
\usepackage{hyperref}
\usepackage[numberedsection=nolabel, toc]{glossaries}
\makeglossary
\newglossaryentry{tex4ht}{name=TeX4ht, description={Konverzní nástroj pro převod \TeX ových dokumentů do jiných formátů, mimo jiné html nebo odt. \href{http://www.tug.org/applications/tex4ht/mn.html}{Homepage}}}
\newglossaryentry{s4ht}{name={Soubory .4ht},text={.4ht}, description={Konfigurační soubory pro LaTeXové balíčky}}
\newglossaryentry{tex4ht-post}{name=tex4ht, 
	description={Postprocesor, který převádí dvi soubor na výstupní xml soubory}}
\newglossaryentry{dvi}{name=dvi,description={}}
\newglossaryentry{make4ht}{name=make4ht,description={}}
\newglossaryentry{xml}{name=XML, description={}}
\newglossaryentry{t4ht}{name=t4ht,
	description={Další postprocesor, který na základě instrukcí předaných tex4ht vytváří CSS soubory a případně provádí konverzi obrázků.}}
%\newglossaryentry{s4ht}{name={Soubory 4ht}, description={Soubory s konfiguracemi balíčků pro TeX4ht}}

\addbibresource{clanek.bib}
\begin{document}
\title{Pracujeme s TeX4ht}
\author{Michal Hoftich}
\maketitle
\tableofcontents
\section{Úvod}
V \href{http://www.abclinuxu.cz/clanky/tex-5-priklad-makra-pro-generovani-html}{článku na Abclinuxu.cz} 
pan \textcite{Olsak13}  popisuje konverzi \TeX ového souboru na html. 
Pod článkem se rozeběhla zajímavá debata nejen o nejlepším způsobu konverze
(La)\TeX{}u, ale také obecně o tom jestli má smysl psát dokumenty v \TeX{}u, nebo spíše v nělterém značkovacím jazyce, ať již je to některý dialekt xml, nebo třeba Markdown.

Protože mám jisté zkušenosti s používáním \gls{tex4ht}%
\footnote{Tvořím jednoduchý build systém \href{https://github.com/michal-h21/make4ht}{make4ht} 
a od něj odvozený \href{https://github.com/michal-h21/tex4ebook}{tex4ebook} 
pro přímou konverzi TeXu do formátů pro čtečky e-booků},
rozhodl jsem se vytvořit krátký článek o jeho používání s češtinou a kódováním utf-8. V článku také ukážu použití některých balíčků, například \verb|biblatex|,
\verb|glossaries|, nebo \verb|minted|.

Zdrojové kódy tohoto článku jsou dostupné na \href{https://github.com/michal-h21/abcclanek}{Githubu}.

\section{Proč TeX4ht?}

\gls{tex4ht} se vyvíjí od druhé poloviny 90. let, v podstatě jako dílo 
jediného autora, Eitana Gurariho. Tento autor bohužel před několika lety zemřel
a v současnosti patrně není nikdo, kdo by se v kódu plně orientoval 
a vývoj mírně řečeno stagnuje. 

Na druhou stranu, v současnosti je nejlepším konvertorem z 
\LaTeX u a poskytuje funkční infrastrukturu pro konfigurací libovolných maker. 
V současnosti je na tvůrcích balíčků nebo jejich uživatelích, aby poskytovali 
tyto konfigurace, není v reálných silách současných správců, aby tvořili konfigurace pro všechny balíčky.

\subsection{Na co konfigurace?}

Bez konfigurací bychom nezískali logické značkování, postprocesor \gls{tex4ht-post} sice vytvoří výstupní soubory, ale ty by obsahovaly jen vizuální formátování získané z dvi souboru, bez jakékoliv logické struktury
(nadpisy, seznamy, odkazy).

Proto \gls{tex4ht} obsahuje mechanismus \gls{s4ht} souborů, které jsou 
automaticky nahrávané spolu s balíčkem téhož jména a kde jsou umístěné 
redefinice některých maker, aby bylo možné je konfigurovat. Více informací 
v sekci \ref{sec:config}

\section{Proč }

\section{Čeština}

Mejme hypotetický článek pro Xe\LaTeX, který se budeme snažit přeložit s \gls{tex4ht}:

\inputminted{latex}{priklady/priklad1.tex}

\gls{tex4ht} obsahuje množství skriptů pro překlad dokumentů. Nejběžnější 
je \verb|htlatex|, který překládá dokument čístým \LaTeX em. Protože máš 
dokument je určený pro Xe\LaTeX, můžeme zkusit variantu tohoto skriptu,
\verb|htxelatex|. 

\begin{minted}{bash}
htxelatex priklad1
\end{minted}

Kompilace se zastaví s chybovou hláškou

\begin{verbatim}
	! LaTeX Error: Command `\acute' already defined in `'.
\end{verbatim}

To nám naznačuje, že něco není v pořádku. Podobných hlášek se objeví více,
můžete zmáčkout klávesu \verb|r| a \verb|enter| pro pokračování kompilace bez 
přerušení. Kompilační skripty překládají dokument \(3 \times\). 
Je to kvůli správnému fungování odkazů, \gls{tex4ht} ukládá při každém průchodu
informace o odkazech do pomocných souborů, tento mechanismus funguje
podobně jako \verb|\label - \ref| mechanismus v \LaTeX{}u.

Po třech průchodech \LaTeX em přichází zpracování výstupního \gls{dvi} 
souboru postprocesorem \gls{tex4ht-post}. Zde vidíme další chybovou hlášku:

\begin{verbatim}
(--- xdv font = [/home/mint/.fonts/LinLibertine_R.otf] (not implemented) ---)
--- warning --- Couldn't find font `[/home/mint/.fonts/LinLibertine_R.otf].ht
\end{verbatim}

Naneštěstí, chyba v postprocesoru \gls{tex4ht-post} znemožňuje použití opentype fontů, pokud si otevřete výsledný html soubor \verb|priklad1.html|, neobjeví se žádný text. Chybové hlášky při překladu \LaTeX em byly 
způsobené balíčkem \verb|fontspec|.  Jak řešit tuto situaci? Máme v zásadě dvě možnosti:

\begin{enumerate}
	\item Upravit náš dokument tak, aby nepoužíval fontspec a opentype fonty
		při běhu \gls{tex4ht}
	\item Použít \href{https://github.com/michal-h21/fontspec}{experimentální verzi} 
		balíčku \verb|fontspec|. 
		\href{http://michal-h21.github.io/fontspec/fontspec-4ht.html}%
		{Ukázkový dokument}.
\end{enumerate}

Druhá možnost  je lehce experimentální, ukážeme si tedy první. 

\gls{tex4ht} se snaží, jak je to jenom možné, aby autor nemusel upravovat 
zdrojové kódy dokumentů. Někdy je to ale jediná možnost, jak zabránit načtení
balíčků, které zprůsobují problémy při kompilaci s \gls{tex4ht}.

Pro zjištění, jestli je dokument kompilován s \gls{tex4ht}, můžeme použít 
konstrukci:
\begin{minted}{latex}
\ifdefined\HCode
   Kod urceny pro TeX4ht
\else
   Kod urceny pro prevod do PDF
\fi
\end{minted}

Makro \verb|\HCode| je definováno \gls{tex4ht} a slouží k vkládání \gls{xml} 
kódů do dokumentu. 

Nyní náš dokument vypadá takto:

\inputminted{latex}{priklady/priklad2.tex}

Místo \verb|htxelatexu| nyní můžeme použít obyčejný \verb|htlatex|,
protože dokument neobsahuje žádná makra vyžadující ke svému běhu Xe\TeX.

\begin{verbatim}
  htlatex priklad2
\end{verbatim}

Text výsledného dokumentu vypadá takto:

%\inputminted[firstline=17, lastline=23]{html}{priklady/priklad2.html}

\begin{verbatim}
P&#x0159;íli&#353; &#382;lutou&#x010D;k&#x00FD; k&#x016F;&#x0148; úp&#x011B;l 
<span class="ecti-1000">&#x010F;</span><span 
class="ecti-1000">ábelsk</span><span 
class="ecti-1000">é </span><span 
class="ecti-1000">ódy</span>. Vyzkou&#353;íme si také ligatury: grafika,
finance, ffi.
\end{verbatim}

Jak je vidět, není výsledek úplně hezký, některé akcentované znaky byly
nahrazené \verb|html| entitami, ostatní jsou v kódování latin-2.

\gls{tex4ht} umí dokument uložit v kódování utf-8, jen je třeba přidat několik
parametrů:

\begin{verbatim}
	htlatex priklad2 "xhtml, charset=utf-8" " -cunihtf -utf8"
\end{verbatim}

Tento ošklivý příkaz se skládá ze tří částí:

\begin{description}
	\item[priklad2] Název souboru
	\item["xhtml, charset=utf-8"] parametry pro baliček \gls{tex4ht}. 
		První parametr označuje document type dokumentu, nebo konfigurační soubor 
		(o těch si povíme více později), 
		\verb|charset=utf-8| určuje, jaké kódování bude uvedenév html souboru. 
	 Tato volba ovšem nezaručuje, že soubor v kódování utf8 skutečně bude!

	 Kompletní seznam možných parametrů je možné najít na 
	 \href{http://www.cvr.cc/?p=504}{blogu CV Radhakrishnana}.
 \item[" -cunihtf -utf8"] parametry pro program \gls{tex4ht-post},
    určují, že bude výsledný soubor uložen v kódování utf8. Všimněte si mezery
    za uvozovkou, musí tam být.
\end{description}		

Text nyní vypadá takto:

\begin{verbatim}
	Příliš žluťoučký kůň úpěl <span 
	class="ecti-1000">ď</span><span 
	class="ecti-1000">ábelsk</span><span 
	class="ecti-1000">é </span><span 
	class="ecti-1000">ódy</span>. Vyzkoušíme si také ligatury: grafika,
	finance, ffi.
\end{verbatim}

Tím jsme vyřešili problém s kódováním znaků, ale stále jsou tu další problémy.
Některá slova jsou rozdělená mezi několik elementů \verb|<span>| a text 
obsahuje ligatury\footnote{To z technických důvodů není ve výpisu vidět, 
	ale můžete se podívat sami: \href{priklady/priklad2.html}{priklad2.html}}.

Tyto dva problémy nění snadné vyřešit čistě jenom v prostředí \gls{tex4ht},
což byl jeden z důvodů, proč jsem vytvořil \href{https://github.com/michal-h21/make4ht}{make4ht}. S pomocí tohoto nástroje můžete jednak zjednodušit 
předávání argumentů \verb|htlatexu|, a můžete využít různé filtry, které 
umožňují vyčistit vygenerované soubory od nadbytečných elementů \verb|<span>|
nebo ligatur. 

Vytvoříme si \verb|make| soubor, se stejným názvem jako \TeX ový dokument, 
jenom s příponou \verb|.mk4|. V našem případě \verb|priklad3.mk4|:

\inputminted{lua}{priklady/priklad3.mk4}

V tomto případě využíváme pouze filtry. Build soubory pro \gls{make4ht} jsou 
skripty v jazyce LUA, takže můžeme využít standardních funkcí jako je 
\verb|require| pro nahrání knihovny \verb|make4ht-filter|. Tato knihovna
vrací funkci, která vytváří další filtrovací funkce. V našem případě 
vytváříme filtrovací funkci \verb|process|, která obsahuje dva filtry -- 
\verb|cleanspan| pro sloučení nadbytečných elementů \verb|<span>| a 
\verb|fixligatures| pro rozdělení ligatur. S pomocí funkce \verb|Make:match|
pak aplikujeme tuto filtrovací funkci na všechny vygenerované soubory s příponou \verb|html|. Další dostupné filtry a funkce \gls{make4ht} jsou popsané
na \href{https://github.com/michal-h21/make4ht}{stránkách projektu}.




Překlad dokumentu je podstatně jednodušší:

\begin{verbatim}
make4ht -u priklad3
\end{verbatim}

A výsledek:

\inputminted[firstline=17, lastline=19]{html}{priklady/priklad3.html}

\section{Konfigurace balíčků}\label{sec:config}


\printglossary
\printbibliography
\end{document}

